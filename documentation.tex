\documentclass[12pt,a4paper]{article}
\usepackage[utf8]{inputenc}
\usepackage[T1]{fontenc}
\usepackage[slovak]{babel}
\usepackage{hyperref}
\usepackage{graphicx}
\usepackage{enumitem}
\usepackage{listings}
\usepackage{xcolor}
\usepackage{geometry}
\usepackage{float}
\usepackage{array}
\usepackage{booktabs}
\usepackage{multirow}
\usepackage{longtable}

% Page geometry
\geometry{
    a4paper,
    top=2.5cm,
    bottom=2.5cm,
    left=2.5cm,
    right=2.5cm
}

% Listings settings for code
\lstset{
    basicstyle=\ttfamily\small,
    breaklines=true,
    frame=single,
    numbers=left,
    numberstyle=\tiny,
    showstringspaces=false,
    keywordstyle=\color{blue},
    commentstyle=\color{green!60!black},
    stringstyle=\color{red},
    literate={á}{{\'a}}1 {é}{{\'e}}1 {í}{{\'i}}1 {ó}{{\'o}}1 {ú}{{\'u}}1
             {ý}{{\'y}}1 {ä}{{\\"a}}1 {ľ}{{\v{l}}}1 {š}{{\v{s}}}1 {č}{{\v{c}}}1
             {ť}{{\v{t}}}1 {ž}{{\v{z}}}1 {ň}{{\v{n}}}1 {ď}{{\v{d}}}1 {ô}{{\^o}}1
}

% Title information
\title{Projekt: Kryptoanalýza klasických šifier pomocou veľkých jazykových modelov\\\large Technická dokumentácia}
\author{}
\date{\today}

\begin{document}

\maketitle

\tableofcontents
\newpage

\section{Anotácia}
Tento dokument predstavuje technickú dokumentáciu projektu zameraného na kryptografickú analýzu klasických šifrovacích metód s využitím veľkých jazykových modelov (LLM). Cieľom projektu je výskum a implementácia inovatívnych kryptoanalytických postupov, ktoré kombinujú tradičné metódy lúštenia šifier s modernými technikami strojového učenia. Vytvorený systém umožňuje dešifrovanie klasických šifier (Cézarova šifra, substitučné a Vigenèrovo šifrovanie, transpozičné šifry a pod.) pomocou LLM modelov, pričom poskytuje webovú aplikáciu na nahrávanie zašifrovaných textov, sledovanie výsledkov dešifrovania a správu používateľov. Dokumentácia popisuje architektúru mikroservisného systému, návrh databázy, integráciu LLM, implementačné detaily, webovú aplikáciu, bezpečnostné mechanizmy, ukážky SQL dotazov, testovanie, históriu vývoja, budúce vylepšenia, nasadenie a údržbu, a záver s zhodnotením prínosov projektu.

\section{Úvod}
\subsection{História kryptoanalýzy}
Kryptografia a kryptoanalýza majú bohatú históriu siahajúcu tisíce rokov do minulosti. Už v staroveku sa používali jednoduché šifrové techniky -- napríklad v Spartskom vojsku transpozičná šifra so skytalou (kožený prúžok omotaný okolo tyče) na maskovanie textu. V 1. storočí pred n.l. zaviedol Julius Caesar jednoduchú substitučnú šifru (tzv. Cézarova šifra), v ktorej sú písmená posunuté o stanovený počet pozícií v abecede. Po stáročia sa vývoj šifier a metód ich lúštenia (kryptoanalýzy) uberal ruka v ruke -- významným medzníkom bol 9. storočie n.l., keď arabský vedec Al-Kindí opísal metódu frekvenčnej analýzy na lámanie monoalfabetických substitučných šifier. Frekvenčná analýza využíva štatistické rozloženie písmen v jazyku a umožnila po prvý raz systematicky lúštiť zašifrované správy bez znalosti kľúča. Ďalší rozvoj priniesli polyalfabetické šifry (napr. Vigenèrova šifra zo 16. storočia), ktoré frekvenčnú analýzu sťažili -- ich prelomenie však umožnili nové postupy (napr. Friedmanov index zhody pre určenie dĺžky kľúča Vigenèrovej šifry). V 20. storočí počas 2. svetovej vojny dosiahol pokrok kryptoanalýzy vrchol v prelomení nemeckej šifry Enigma, na čom sa podieľali matematické postupy aj prvé elektromechanické počítače. História teda ukazuje neustály súboj medzi tvorcami šifier a kryptoanalytikmi -- každá nová šifra viedla k vynálezu nových metód na jej prelomenie.

\subsection{Prečo LLM v kryptoanalýze}
V súčasnosti do tejto oblasti vstupujú metódy umelej inteligencie a strojového učenia. Veľké jazykové modely (LLM) ako GPT-3 alebo GPT-4 preukázali schopnosť analyzovať a generovať prirodzený jazyk, a otvára sa otázka, do akej miery dokážu pomôcť pri lúštení zašifrovaných textov. Klasická kryptoanalýza je často formulovaná ako problém hľadania správneho kľúča alebo princípu šifry pomocou matematických a štatistických metód. LLM na to idú z iného smeru -- ako modely jazyka sa snažia predpovedať najpravdepodobnejšie slová a vety, ktoré by mohli byť v texte. Tým pádom môžu využiť svoje znalosti o štruktúre jazyka na odhalenie vzorcov v šifrovanom texte.

\section{Architektúra systému}
Systém je navrhnutý ako moderná webová aplikácia s mikroservisnou architektúrou. Hlavné komponenty zahŕňajú:

\begin{itemize}
    \item Webový server (Flask) -- spracováva HTTP požiadavky, autentifikáciu a biznis logiku
    \item Databáza (PostgreSQL) -- ukladá údaje o používateľoch, pokusoch a výsledkoch
    \item LLM modul -- obsahuje modely a logiku pre dešifrovanie
    \item Webové rozhranie -- poskytuje používateľské rozhranie (HTML, CSS, JavaScript)
\end{itemize}

\section{Návrh databázy}
\subsection{Konceptuálny model}
Návrh databázy prešiel troma úrovňami: konceptuálny model, logický model a fyzický model. V tejto sekcii najprv popíšeme konceptuálny ER model a entity, potom logický model s atribútmi a väzbami a nakoniec fyzickú realizáciu v PostgreSQL vrátane schém tabuliek.

\subsubsection{Entity}
\begin{itemize}
    \item \textbf{Šifry} -- reprezentujú druhy klasických šifier, ktoré systém eviduje a vie analyzovať. Každá šifra má:
    \begin{itemize}
        \item Identifikátor
        \item Názov
        \item Historické obdobie pôvodu
        \item Krajina/miesto vzniku
        \item Princíp šifrovania
        \item Ukážkový zašifrovaný text a plaintext
    \end{itemize}
    
    \item \textbf{Modely} -- predstavujú LLM modely použité na dešifrovanie:
    \begin{itemize}
        \item Identifikátor
        \item Názov modelu
        \item Zameranie/typ
        \item Verzia
    \end{itemize}
    
    \item \textbf{Pokusy o dešifrovanie} -- každá požiadavka používateľa:
    \begin{itemize}
        \item ID pokusu
        \item Referencia na používateľa
        \item Referencia na šifru a model
        \item Čas začiatku a konca
        \item Indikátor úspechu
        \item Percento správnosti
        \item Zašifrovaný a dešifrovaný text
    \end{itemize}
    
    \item \textbf{Výsledky dešifrovania} -- detailné výstupy:
    \begin{itemize}
        \item ID výsledku
        \item Referencia na pokus
        \item Vygenerovaný text
        \item Miera podobnosti
        \item Čitateľnosť výstupu
    \end{itemize}
    
    \item \textbf{Manuálne korekcie} -- ručné opravy výstupov:
    \begin{itemize}
        \item ID korekcie
        \item Referencia na výsledok
        \item Korektor (používateľ)
        \item Percento zmien
        \item Finálny text
    \end{itemize}
    
    \item \textbf{Používatelia} -- registrovaní používatelia:
    \begin{itemize}
        \item ID používateľa
        \item Používateľské meno
        \item E-mail
        \item Hash hesla
        \item Dátum vytvorenia účtu
        \item OAuth identifikátor
    \end{itemize}
\end{itemize}

\subsubsection{Vzťahy medzi entitami}
\begin{itemize}
    \item Používateľ 1:N Pokusy o dešifrovanie
    \item Šifry 1:N Pokusy
    \item Modely 1:N Pokusy
    \item Pokusy 1:1/1:N Výsledky
    \item Výsledky 1:N Manuálne korekcie
\end{itemize}

\subsection{Logický model}
Logický model upresňuje detailnú štruktúru databázy. Pre implementáciu sme zvolili relačný databázový model v PostgreSQL.

\subsubsection{Tabuľky a ich štruktúra}
\begin{lstlisting}[language=SQL]
CREATE TABLE "Users" (
    user_id SERIAL PRIMARY KEY,
    username VARCHAR(50) UNIQUE NOT NULL,
    email VARCHAR(100) UNIQUE NOT NULL,
    password_hash VARCHAR(256) NOT NULL,
    created_at TIMESTAMP DEFAULT CURRENT_TIMESTAMP
);

CREATE TABLE "Ciphers" (
    cipher_id SERIAL PRIMARY KEY,
    name VARCHAR(100) UNIQUE NOT NULL,
    historical_period VARCHAR(50),
    origin VARCHAR(50),
    encryption_principles TEXT,
    encrypted_text TEXT,
    plaintext TEXT
);

CREATE TABLE "Models" (
    model_id SERIAL PRIMARY KEY,
    name VARCHAR(100) UNIQUE NOT NULL,
    type VARCHAR(50) NOT NULL,
    version VARCHAR(20) NOT NULL
);

CREATE TABLE "Decryption_Attempts" (
    attempt_id SERIAL PRIMARY KEY,
    user_id INTEGER REFERENCES "Users"(user_id),
    cipher_id INTEGER REFERENCES "Ciphers"(cipher_id),
    model_id INTEGER REFERENCES "Models"(model_id),
    start_time TIMESTAMP NOT NULL,
    end_time TIMESTAMP NOT NULL,
    success BOOLEAN NOT NULL,
    correctness_percentage DECIMAL(5,2),
    encrypted_text TEXT NOT NULL,
    decrypted_text TEXT
);

CREATE TABLE "Decryption_Results" (
    result_id SERIAL PRIMARY KEY,
    attempt_id INTEGER REFERENCES "Decryption_Attempts"(attempt_id),
    model_output TEXT NOT NULL,
    similarity_measure DECIMAL(5,2),
    readability_level DECIMAL(5,2)
);

CREATE TABLE "Manual_Corrections" (
    correction_id SERIAL PRIMARY KEY,
    result_id INTEGER REFERENCES "Decryption_Results"(result_id),
    corrector INTEGER REFERENCES "Users"(user_id),
    changed_percentage DECIMAL(5,2),
    final_text TEXT NOT NULL
);
\end{lstlisting}

\section{Implementačné detaily}
\subsection{Server a API endpointy}
Server je implementovaný v jazyku Python s využitím frameworku Flask. Hlavné endpointy zahŕňajú:

\begin{itemize}
    \item \texttt{/register}, \texttt{/login}, \texttt{/logout} -- správa používateľov
    \item \texttt{/decrypt} -- spracovanie požiadaviek na dešifrovanie
    \item \texttt{/attempts} -- história pokusov
    \item \texttt{/correct} -- uloženie manuálnych korekcií
\end{itemize}

\subsection{Integrácia LLM}
Pre prácu s jazykovými modelmi využívame knižnicu \texttt{transformers} od Hugging Face. Hlavné komponenty:

\begin{itemize}
    \item Načítanie modelu: Používame model DistilGPT-2 pre jeho dobrý pomer výkon/veľkosť
    \item Tokenizácia: Vstupný text je tokenizovaný pomocou GPT-2 tokenizéra
    \item Generovanie: Model generuje najpravdepodobnejší otvorený text
\end{itemize}

\subsection{Databázové operácie}
Komunikácia s PostgreSQL databázou je realizovaná cez knižnicu \texttt{psycopg2}. Príklad SQL dotazu pre overenie prihlasovacích údajov:

\begin{lstlisting}[language=SQL]
SELECT user_id, username, email, password_hash
FROM "Users"
WHERE email = %s;
\end{lstlisting}

\section{Webová aplikácia}
\subsection{Front-end a používateľské funkcie}
Webová aplikácia poskytuje používateľsky prívetivé rozhranie na využívanie funkcionalít systému. Je navrhnutá tak, aby zjednodušila prácu kryptoanalytika.

\subsection{Architektúra front-endu}
Aplikácia využíva:
\begin{itemize}
    \item Serverom renderované HTML šablóny (Flask \texttt{render\_template})
    \item Dynamické načítavanie dát cez JavaScript (AJAX)
    \item CSS framework Bulma pre moderný vzhľad
    \item Material Icons pre zobrazenie ikoniek
\end{itemize}

\subsection{Autentifikácia a správa používateľov}
\begin{itemize}
    \item Prihlasovanie stráži Flask-Login
    \item Registrácia nového účtu vyžaduje meno, e-mail a heslo
    \item Integrovaná možnosť OAuth prihlásenia cez Google
    \item Heslá sú hashované algoritmom PBKDF2
\end{itemize}

\subsection{Hlavná stránka -- Dešifrovanie}
Hlavné rozhranie je rozdelené do dvoch stĺpcov:
\begin{itemize}
    \item Ľavý bočný panel
    \begin{itemize}
        \item Navigačné tlačidlá
        \item Timeline s históriou pokusov
        \item Možnosti nastavení
    \end{itemize}
    \item Pravý obsahový panel
    \begin{itemize}
        \item Formulár na dešifrovanie textu
        \item Výber šifry a modelu
        \item Zobrazenie výsledkov
    \end{itemize}
\end{itemize}

\subsection{Tabulkové údaje}
Zobrazuje históriu dešifrovaní vo forme tabuľky s nasledujúcimi stĺpcami:

\begin{longtable}{p{0.1\textwidth}p{0.1\textwidth}p{0.1\textwidth}p{0.1\textwidth}p{0.1\textwidth}p{0.1\textwidth}p{0.15\textwidth}p{0.15\textwidth}}
\toprule
\textbf{ID} & \textbf{Šifra} & \textbf{Model} & \textbf{Začiatok} & \textbf{Koniec} & \textbf{Úspech} & \textbf{Zašifrovaný text} & \textbf{Dešifrovaný text} \\
\midrule
\endhead
1 & Cézarova & GPT-2 & 2024-03-14 10:00 & 2024-03-14 10:01 & 95\% & "Khoor zruog" & "Hello world" \\
2 & Vigenère & GPT-3 & 2024-03-14 10:15 & 2024-03-14 10:16 & 87\% & "Wnqqs iptqf" & "Hello world" \\
\bottomrule
\end{longtable}

Tabuľka obsahuje kompletný prehľad všetkých pokusov o dešifrovanie, vrátane:
\begin{itemize}
    \item Časových značiek začiatku a konca
    \item Použitého modelu a typu šifry
    \item Percentuálnej úspešnosti dešifrovania
    \item Vstupného a výstupného textu
\end{itemize}

\section{Bezpečnostné opatrenia}
Pri návrhu systému sme kládli dôraz na to, aby bol bezpečný, najmä keď pracuje s používateľskými účtami a potenciálne citlivými dátami. Kombinujeme viacero vrstiev ochrany:

\subsection{Hašovanie hesiel}
Heslá používateľov sa v databáze nikdy neukladajú v čitateľnej forme. Využívame funkcie \texttt{generate\_password\_hash()} a \texttt{check\_password\_hash()} z Flask/Werkzeug, ktoré implementujú silné jednosmerné hashovanie hesiel (predvolene PBKDF2-SHA256 so soľou).

\subsection{OAuth 2.0 integrácia}
Pre zvýšenie bezpečnosti a komfortu majú používatelia možnosť využiť externého poskytovateľa identity -- Google. OAuth 2.0 implementujeme pomocou knižnice Authlib, ktorá sa stará o celú výmenu tokenov.

\subsection{Bezpečné pripojenie k databáze}
\begin{itemize}
    \item Komunikácia cez zabezpečené SSL spojenie
    \item Prístup chránený heslom a obmedzenými právami
    \item Tajnosti uložené v súbore \texttt{.env}
\end{itemize}

\subsection{Ochrana proti SQL injection}
Všetky databázové operácie realizujeme pomocou parametrov v SQL dotazoch, nie skladaním reťazcov. Napríklad:

\begin{lstlisting}[language=SQL]
cur.execute("SELECT ... WHERE email=%s", (email,))
\end{lstlisting}

\subsection{Rate limiting}
Obmedzenie počtu požiadaviek na:
\begin{itemize}
    \item Max 5 dešifrovaní za minútu na používateľa
    \item Max 20 requestov/min z jednej IP
\end{itemize}

\subsection{XSS a ochrana na strane klienta}
\begin{itemize}
    \item Bezpečné zobrazovanie údajov v HTML
    \item Escapovanie špeciálnych znakov
    \item Používanie textNode alebo innerText namiesto innerHTML
\end{itemize}

\section{Ukážky dôležitých SQL dotazov}
V tejto časti uvádzame niekoľko kľúčových SQL dotazov použitých v aplikácii, spolu s vysvetlením ich funkcie.

\subsection{Overenie prihlasovacích údajov používateľa}
\begin{lstlisting}[language=SQL]
SELECT user_id, username, email, password_hash
FROM "Users"
WHERE email = %s;
\end{lstlisting}

Tento dotaz vyberá riadok používateľa podľa e-mailovej adresy. Aplikácia následne porovná hash hesla s tým, čo používateľ zadal.

\subsection{Vloženie nového pokusu o dešifrovanie}
\begin{lstlisting}[language=SQL]
INSERT INTO "Decryption_Attempts"
(cipher_id, model_id, user_id, start_time, end_time, success,
correctness_percentage, encrypted_text, decrypted_text)
VALUES (%s, %s, %s, %s, %s, %s, %s, %s, %s)
RETURNING attempt_id;
\end{lstlisting}

Tento dotaz vloží nový riadok do tabuľky pokusov so všetkými detailmi.

\subsection{Výber histórie dešifrovacích pokusov}
\begin{lstlisting}[language=SQL]
SELECT da.attempt_id,
       c.name AS cipher_name,
       m.name AS model_name,
       da.start_time,
       da.end_time,
       da.success,
       da.correctness_percentage,
       da.encrypted_text,
       da.decrypted_text,
       dr.result_id,
       dr.model_output,
       dr.similarity_measure,
       dr.readability_level,
       mc.correction_id,
       mc.corrector,
       mc.changed_percentage,
       mc.final_text
FROM "Decryption_Attempts" da
JOIN "Ciphers" c ON da.cipher_id = c.cipher_id
JOIN "Models" m ON da.model_id = m.model_id
LEFT JOIN "Decryption_Results" dr ON da.attempt_id = dr.attempt_id
LEFT JOIN "Manual_Corrections" mc ON dr.result_id = mc.result_id
WHERE da.user_id = %s
ORDER BY da.start_time DESC;
\end{lstlisting}

Tento komplexný dotaz spája viacero tabuliek pre zobrazenie kompletnej histórie pokusov.

\subsection{Uloženie manuálnej korekcie výsledku}
\begin{lstlisting}[language=SQL]
INSERT INTO "Manual_Corrections"
(result_id, corrector, changed_percentage, final_text)
VALUES (%s, %s, %s, %s);
\end{lstlisting}

Tento dotaz pridá nový záznam do tabuľky manuálnych korekcií.

\section{Testovanie}
Testovanie systému prebiehalo na viacerých úrovniach, aby sme zabezpečili spoľahlivosť a kvalitu implementácie.

\subsection{Jednotkové testy}
Implementovali sme sadu jednotkových testov pre kľúčové komponenty:
\begin{itemize}
    \item Testy šifrovacích/dešifrovacích funkcií
    \item Testy autentifikácie a správy používateľov
    \item Testy databázových operácií
    \item Testy integrácie s LLM
\end{itemize}

\subsection{Integračné testy}
Overenie spolupráce komponentov:
\begin{itemize}
    \item End-to-end testy webového rozhrania
    \item Testy komunikácie medzi serverom a databázou
    \item Testy správneho fungovania celého procesu dešifrovania
\end{itemize}

\section{História vývoja}
\subsection{Analýza histórie commitov}
Vývoj projektu prebiehal v nasledujúcich hlavných fázach:

\begin{enumerate}
    \item \textbf{Inicializácia projektu} (Commit: Initial commit)
    \begin{itemize}
        \item Vytvorenie základnej štruktúry projektu
        \item Nastavenie Flask aplikácie
        \item Inicializácia databázy
    \end{itemize}
    
    \item \textbf{Implementácia databázovej vrstvy} (Commit: Database implementation)
    \begin{itemize}
        \item Vytvorenie SQL schém
        \item Implementácia modelov
        \item Migrácie databázy
    \end{itemize}
    
    \item \textbf{Integrácia LLM} (Commit: LLM integration)
    \begin{itemize}
        \item Pridanie DistilGPT-2 modelu
        \item Implementácia dešifrovacieho modulu
        \item Optimalizácia výkonu
    \end{itemize}
    
    \item \textbf{Vývoj webového rozhrania} (Commit: Web UI development)
    \begin{itemize}
        \item Implementácia používateľského rozhrania
        \item Pridanie autentifikácie
        \item Vylepšenie UX
    \end{itemize}
    
    \item \textbf{Testovanie a ladenie} (Commit: Testing and optimization)
    \begin{itemize}
        \item Jednotkové testy
        \item Integračné testy
        \item Výkonnostné testy
    \end{itemize}
    
    \item \textbf{Dokumentácia a finalizácia} (Commit: Documentation and cleanup)
    \begin{itemize}
        \item Technická dokumentácia
        \item Používateľská príručka
        \item Čistenie kódu
    \end{itemize}
\end{enumerate}

\section{Budúce vylepšenia}
Na základe analýzy súčasného stavu projektu navrhujeme nasledujúce vylepšenia:

\subsection{Rozšírenie podporovaných šifier}
\begin{itemize}
    \item Implementácia ďalších historických šifier:
    \begin{itemize}
        \item Playfairova šifra
        \item Hillova šifra
        \item ADFGVX šifra
    \end{itemize}
    \item Podpora moderných šifrovacích algoritmov
    \item Rozšírenie databázy ukážkových textov
\end{itemize}

\subsection{Vylepšenie modelov}
\begin{itemize}
    \item Fine-tuning modelov na špecifické šifry
    \item Implementácia ensemble prístupu
    \item Experimentovanie s novšími architektúrami
    \item Optimalizácia hyperparametrov
\end{itemize}

\subsection{Vylepšenie používateľského rozhrania}
\begin{itemize}
    \item Interaktívna vizualizácia procesu dešifrovania
    \item Real-time aktualizácie stavu
    \item Mobilná verzia aplikácie
    \item Rozšírené štatistiky a reporty
\end{itemize}

\subsection{Optimalizácia výkonu}
\begin{itemize}
    \item Implementácia cachingu
    \item Distribuované spracovanie
    \item Optimalizácia databázových dotazov
    \item Škálovanie na cloud platformy
\end{itemize}

\subsection{Bezpečnostné vylepšenia}
\begin{itemize}
    \item Implementácia 2FA
    \item Audit logov
    \item Pravidelné bezpečnostné audity
    \item Vylepšená ochrana proti útokom
\end{itemize}

\subsection{Rozšírenie funkcionality}
\begin{itemize}
    \item API pre externé aplikácie
    \item Podpora viacerých jazykov
    \item Export/import funkcionalita
    \item Integrácia s inými nástrojmi
\end{itemize}

\section{Nasadenie a údržba}
\subsection{Docker kontajnerizácia}
Aplikácia je zabalená do Docker kontajnera pre jednoduché nasadenie:
\begin{itemize}
    \item Multi-stage Dockerfile
    \item Docker Compose pre orchestráciu služieb
    \item Automatické buildy a nasadenie
\end{itemize}

\subsection{CI/CD pipeline}
Využívame GitHub Actions na automatizáciu:
\begin{itemize}
    \item Spustenie testov a lintera
    \item Build a push Docker image
    \item Automatické nasadenie na produkčný server
\end{itemize}

\subsection{Zálohovanie dát}
Stratégia zálohovania:
\begin{itemize}
    \item Nočný PostgreSQL dump
    \item Uchovávanie záloh v cloud storage
    \item Zálohovanie natrénovaných modelov
\end{itemize}

\subsection{Monitorovanie a logovanie}
Implementované nástroje:
\begin{itemize}
    \item Logging do stdout
    \item Health-check endpoint
    \item Prometheus/Grafana monitoring
    \item Error tracking cez Sentry
\end{itemize}

\section{Záver}
Projekt Kryptoanalýza klasických šifier pomocou veľkých jazykových modelov ukázal, že prepojenie tradičnej kryptografie s modernými metódami umelej inteligencie je nielen možné, ale aj veľmi perspektívne. Vytvorili sme funkčný systém, ktorý dokáže za pomoci LLM modelu (GPT-2) automaticky dešifrovať jednoduché šifry a učiť sa z príkladov. Systém pozostáva z prehľadnej webovej aplikácie pre používateľov, robustného serverového backendu postaveného na Flasku, a výkonného databázového úložiska PostgreSQL, ktoré spolu tvoria ucelenú platformu pre experimentovanie s kryptoanalýzou.

Hlavným prínosom projektu je inovatívny prístup ku kryptoanalýze: namiesto ručnej frekvenčnej analýzy či brute-force sme využili znalosti zachytené v jazykovom modeli. Model bol schopný rozoznať vzory v zašifrovanom texte a v mnohých prípadoch poskytnúť správny alebo aspoň čiastočne správny otvorený text. To potvrdzuje hypotézu, že LLM (tréningom na obrovských množstvách textu) v sebe implicitne nesú informácie použiteľné pri lúštení šifier -- vedia generovať zmysluplný text aj z nezrozumiteľných vstupov.

Zároveň sme identifikovali obmedzenia: komplexnejšie klasické šifry (polyalfabetické s dlhým kľúčom, transpozičné so zamotanou permutáciou) kladú na LLM vyššie nároky a samotný generatívny prístup nemusí stačiť. V týchto prípadoch by bolo vhodné kombinovať AI s tradičnými algoritmickými metódami. Náš systém je modulárny, takže takú kombináciu by bolo možné v budúcnosti doplniť.

Z pohľadu praktických výsledkov systém poskytuje:
\begin{itemize}
    \item Rýchle automatizované dešifrovanie pre bežné školské/praktické príklady šifier
    \item Platformu pre ďalší výskum -- možnosť zapojiť nové modely alebo techniky
    \item Dôkaz konceptu, že LLM vedia riešiť štruktúrované problémy
\end{itemize}

Na záver možno konštatovať, že projekt splnil stanovené ciele: vytvorili sme funkčnú implementáciu, zdokumentovali architektúru, a v testoch sme dosiahli zaujímavé výsledky. Systém je pripravený pre ďalšie používanie aj rozširovanie. Kombinácia klasickej kryptografie a LLM je stále pomerne nová oblasť a náš projekt k nej prispieva konkrétnym riešením.

\section{Zoznam literatúry}
\begin{enumerate}
    \item IBM -- The History of Cryptography. Think Blog by IBM, 2021. (História kryptografie od staroveku po súčasnosť, vrátane zmienky o Al-Kindím a frekvenčnej analýze)

    \item Maskey, U., Zhu, C., Naseem, U. -- "Benchmarking Large Language Models for Cryptanalysis and Mismatched-Generalization". arXiv preprint arXiv:2505.24621, May 2025. (Výskumný článok hodnotiaci schopnosti viacerých LLM dešifrovať rôzne šifry v rôznych podmienkach)

    \item Sugio, N. -- "Implementation of Cryptanalytic Programs Using ChatGPT". IACR Cryptology ePrint Archive, Report 2024/240, 2024. (Štúdia demonštrujúca použitie pokročilého LLM ChatGPT na generovanie zdrojového kódu pre kryptanalýzu moderných šifier, vrátane diskusie potenciálu a obmedzení AI v kryptografii)

    \item Reddit -- GPT-4 can break encryption (Caesar Cipher). príspevok užívateľa himey72 v subreddite r/ChatGPT, apríl 2023. (Diskusia potvrdzujúca, že GPT-4 si poradí s Cézarovou šifrou, no s inými šiframi má problémy)

    \item CEUR Workshop Proceedings -- "Neural Cryptanalysis of Classical Ciphers". HistoCrypt 2018. (Práca prezentujúca metódu využitia neurónových sietí v kombinácii s klasickými technikami kryptoanalýzy na automatizovanie lúštenia klasických šifier)
\end{enumerate}

\section{Integrácia veľkých jazykových modelov}
\subsection{Výber a implementácia modelov}
Pre kryptoanalýzu sme zvolili model DistilGPT-2 z nasledujúcich dôvodov:
\begin{itemize}
    \item Dobrý pomer výkon/veľkosť -- model je dostatočne výkonný pre naše potreby, ale zároveň nie je príliš náročný na zdroje
    \item Možnosť lokálneho behu -- nie je potrebné volať externé API
    \item Podpora knižnice transformers -- jednoduchá integrácia a fine-tuning
    \item Vhodnosť pre generatívne úlohy -- model je trénovaný na predikciu nasledujúcich tokenov
\end{itemize}

\subsection{Architektúra dešifrovacieho modulu}
Dešifrovací modul je implementovaný v triede Decryptor s nasledujúcou štruktúrou:

\begin{lstlisting}[language=Python]
class Decryptor:
    def __init__(self, model_name="distilgpt2"):
        self.tokenizer = AutoTokenizer.from_pretrained(model_name)
        self.model = AutoModelForCausalLM.from_pretrained(model_name)
        
    def decrypt(self, encrypted_text, max_length=100):
        # Tokenizácia vstupu
        inputs = self.tokenizer(encrypted_text, return_tensors="pt")
        
        # Generovanie výstupu
        outputs = self.model.generate(
            inputs["input_ids"],
            max_length=max_length,
            num_return_sequences=1,
            no_repeat_ngram_size=2,
            do_sample=True,
            top_k=50,
            top_p=0.95,
            temperature=0.7
        )
        
        # Dekódovanie výstupu
        decrypted_text = self.tokenizer.decode(outputs[0])
        return decrypted_text
\end{lstlisting}

\subsection{Proces dešifrovania}
Proces dešifrovania prebieha v nasledujúcich krokoch:

\begin{enumerate}
    \item \textbf{Príprava vstupu}:
    \begin{itemize}
        \item Zašifrovaný text je tokenizovaný pomocou GPT-2 tokenizéra
        \item Pridanie špeciálnych tokenov a padding
        \item Konverzia na tenzory PyTorch
    \end{itemize}
    
    \item \textbf{Generovanie}:
    \begin{itemize}
        \item Model generuje najpravdepodobnejšie tokeny
        \item Použitie beam search alebo sampling stratégií
        \item Aplikácia parametrov ako temperature a top-k/top-p
    \end{itemize}
    
    \item \textbf{Post-processing}:
    \begin{itemize}
        \item Dekódovanie tokenov späť na text
        \item Odstránenie špeciálnych tokenov
        \item Validácia výstupu
    \end{itemize}
\end{enumerate}

\subsection{Metriky a vyhodnotenie}
Pre vyhodnotenie kvality dešifrovania používame nasledujúce metriky:

\begin{itemize}
    \item \textbf{Charakterová presnosť}:
    \begin{itemize}
        \item Porovnanie vygenerovaného textu so skutočným plaintextom znak po znaku
        \item Ignorovanie veľkosti písmen a medzier
        \item Výpočet percentuálnej zhody pomocou SequenceMatcher
    \end{itemize}
    
    \item \textbf{Čitateľnosť}:
    \begin{itemize}
        \item Percento slov vo výstupe, ktoré sú reálne slová
        \item Kontrola proti slovníku alebo jazykovému modelu
        \item Hodnotenie plynulosti a gramatickej správnosti
    \end{itemize}
    
    \item \textbf{Sémantická podobnosť}:
    \begin{itemize}
        \item Porovnanie významu dešifrovaného textu s originálom
        \item Využitie embeddings alebo BLEU skóre
        \item Subjektívne hodnotenie kontextu
    \end{itemize}
\end{itemize}

\subsection{Optimalizácia výkonu}
Pre zlepšenie výkonu modelu sme implementovali:

\begin{itemize}
    \item Caching modelov v pamäti
    \item Batch processing pre viacero požiadaviek
    \item Optimalizácia hyperparametrov generovania
    \item Možnosť paralelného spracovania na GPU
\end{itemize}

\end{document} 
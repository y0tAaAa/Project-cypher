% Acronyms
% ========
%
% An acronym is a word formed from the initial letters in a phrase. 
%
% Acronym Definition Exapmle:
% ---------------------------
% \newacronym{gcd}{GCD}{Greatest Common Divisor}
% \newacronym{dry}{DRY}{Don't Repeat Yourself}
%
% Usage:
% ------
% You can use these three options:
% 
% \acrlong{}  
%   Displays the phrase which the acronyms stands for. Put the label of the acronym inside the braces. In the example, \acrlong{gcd} prints Greatest Common Divisor. 
%
% \acrshort{} 
%   Prints the acronym whose label is passed as parameter. For instance, \acrshort{gcd} renders as GCD. 
%
% \acrfull{ } 
%   Prints both, the acronym and its definition. In the example the output of \acrfull{dry} is Don't Repeat Yourself (DRY). 
% 
% For more information see:
% -------------------------
% * https://www.sharelatex.com/learn/Glossaries 
% * https://en.wikibooks.org/wiki/LaTeX/Glossary
%

\chapter*{Zoznam skratiek}
\addcontentsline{toc}{chapter}{Zoznam skratiek}

\begin{acronym}[WSXXXX]
    \acro{LLM}[LLM]{Large Language Model}
    \acro{API}[API]{Application Programming Interface}
    \acro{REST}[REST]{Representational State Transfer}
    \acro{SQL}[SQL]{Structured Query Language}
    \acro{GPU}[GPU]{Graphics Processing Unit}
    \acro{SSL}[SSL]{Secure Sockets Layer}
    \acro{TLS}[TLS]{Transport Layer Security}
    \acro{OAuth}[OAuth]{Open Authorization}
    \acro{XSS}[XSS]{Cross-Site Scripting}
    \acro{ML}[ML]{Machine Learning}
    \acro{UI}[UI]{User Interface}
    \acro{HTTP}[HTTP]{Hypertext Transfer Protocol}
    \acro{JSON}[JSON]{JavaScript Object Notation}
    \acro{CPU}[CPU]{Central Processing Unit}
    \acro{RAM}[RAM]{Random Access Memory}
\end{acronym}

%% -----------------------------------------------------------------
%% This file uses UTF-8 encoding
%%
%% For compilation use following command:
%% latexmk -pdf -pvc -bibtex thesis
%%
%% -----------------------------------------------------------------
%%                                     _     _      
%%      _ __  _ __ ___  __ _ _ __ ___ | |__ | | ___ 
%%     | '_ \| '__/ _ \/ _` | '_ ` _ \| '_ \| |/ _ \
%%     | |_) | | |  __/ (_| | | | | | | |_) | |  __/
%%     | .__/|_|  \___|\__,_|_| |_| |_|_.__/|_|\___|
%%     |_|                                          
%%
%% -----------------------------------------------------------------
\documentclass{kithesis}
% Additional packages
\usepackage[main=slovak,english]{babel}
%\usepackage[pdftex]{graphicx}
\usepackage{epstopdf}
\DeclareGraphicsExtensions{.eps}
\DeclareGraphicsExtensions{.pdf,.png,.jpg,.mps}
\graphicspath{{figures/}} % priecinok na obrazky
\usepackage{parskip}% 'zhusti' polozky obsahu
\usepackage{pdfpages}
\usepackage[euler]{textgreek} % pre pisanie greckych pismen priamo v texte
\usepackage{mathtools}
\usepackage{amsmath,amsfonts,amssymb}
\usepackage{upgreek} % napr. $\upmu\mathrm{m}$ pre mikrometer ...
\usepackage{listings}  % for source code
\usepackage[noprefix]{nomencl}
\makeglossary % prikaz na vytvorenie suboru .glo
\usepackage{biblatex}
\def\hyph{-\penalty0\hskip0pt\relax} %% toto je pre nedelitelnost slov, napr. M\hyph postupnosti - potom sa nerozdeli slovo na konci riadku
\usepackage{textcomp} %na to aby som vedel pisat °
\usepackage{nicefrac} % pekne zlomky
\usepackage[justification=centering,caption=false,font=footnotesize]{subfig}
\usepackage{multirow}
\usepackage{relsize}
\usepackage{bbding}
\usepackage{threeparttable}
\usepackage{xcolor}
\usepackage{acronym}

\usepackage{scrextend}
\deffootnote[10pt]{10pt}{1pt}{\textsuperscript{\thefootnotemark}\,}


% Listings settings
% See for details: https://en.wikibooks.org/wiki/LaTeX/Source_Code_Listings
\lstset{
    basicstyle=\small\ttfamily,  % smaller typewriter font
    showstringspaces=false       % don't show spaces in string
}
\def\lstlistingname{Zdrojový kód}
% Variables
%\thesisspec{figures/thesisspec.png} 0

% definicnia premennej pre to aby sa nezobrzovalo cilo v zozname vybranej literatury %%
%%%%%%%%%%%%%%%%%%%%%%%%%%%%%%%%%%%%%%%%%%%%%%%%%%%%%%%%%%%%%%
\defbibenvironment{mypubs}
{\list
	{}
	{\setlength{\leftmargin}{\bibhang}%
	    %		\setlength{\itemindent}{\leftmargin}%
				\setlength{\itemsep}{\bibitemsep}%
	\setlength{\parsep}{\bibparsep}}}
{\endlist}
{\item}
%%%%%%%%%%%%%%%%%%%%%%%%%%%%%%%%%%%%%%%%%%%%%%%%%%%%%%%%%%%%%%



\title{Kryptoanalýza pomocou veľkých jazykových modelov}{Cryptanalysis using Large Language Models}


\author{Meno}{Priezvisko}
\supervisor{Ing. Zuzana Sokolová, PhD.} %veduci prace
%\consultant{Ing. Zuzana Sokolová, PhD.} %konzultant
%\college{University of Žilina}{Žilinská univerzita} %univerzita
%\faculty{Faculty of Electrical Engineering and informatics}{Fakulta elektrotechniky a informatiky} %fakulta

%\thesis{Master thesis}{Diplomová práca} %typ prace
\submissiondate{01}{05}{2025}
\fieldofstudy{informatika}
\studyprogramme{počítačové siete}
%\city{Košice} %mesto
\keywords{kryptoanalýza, strojové učenie, jazykové modely, šifrovanie}{cryptanalysis, machine learning, language models, encryption}
%\declaration{som nepodvadzal}

\abstract{
This project investigates the capabilities of neural network models (Large Language Models - LLMs) for cryptanalysis of classical ciphers. The system combines traditional cryptographic methods with modern machine learning approaches to create an efficient tool for decryption and analysis. The implementation includes a web interface, REST API, and integration with state-of-the-art language models.

The project demonstrates the potential of using large language models in cryptanalysis and opens new possibilities for future research in this field. The system is designed with emphasis on scalability, security, and user experience.
}{
Tento projekt skúma možnosti neurónových sieťových modelov (Large Language Models - LLMs) pre kryptoanalýzu klasických šifier. Systém kombinuje tradičné kryptografické metódy s modernými prístupmi strojového učenia na vytvorenie efektívneho nástroja pre dešifrovanie a analýzu. Implementácia zahŕňa webové rozhranie, REST API a integráciu s modernými jazykovými modelmi.

Projekt demonštruje potenciál využitia veľkých jazykových modelov v oblasti kryptoanalýzy a otvára nové možnosti pre budúci výskum v tejto oblasti. Systém je navrhnutý s dôrazom na škálovateľnosť, bezpečnosť a používateľskú prívetivosť.
}

\acknowledgment{Poďakovanie za podporu pri vývoji projektu a poskytnutie potrebných zdrojov.}


% Load acronyms
% Acronyms
% ========
%
% An acronym is a word formed from the initial letters in a phrase. 
%
% Acronym Definition Exapmle:
% ---------------------------
% \newacronym{gcd}{GCD}{Greatest Common Divisor}
% \newacronym{dry}{DRY}{Don't Repeat Yourself}
%
% Usage:
% ------
% You can use these three options:
% 
% \acrlong{}  
%   Displays the phrase which the acronyms stands for. Put the label of the acronym inside the braces. In the example, \acrlong{gcd} prints Greatest Common Divisor. 
%
% \acrshort{} 
%   Prints the acronym whose label is passed as parameter. For instance, \acrshort{gcd} renders as GCD. 
%
% \acrfull{ } 
%   Prints both, the acronym and its definition. In the example the output of \acrfull{dry} is Don't Repeat Yourself (DRY). 
% 
% For more information see:
% -------------------------
% * https://www.sharelatex.com/learn/Glossaries 
% * https://en.wikibooks.org/wiki/LaTeX/Glossary
%

\chapter*{Zoznam skratiek}
\addcontentsline{toc}{chapter}{Zoznam skratiek}

\begin{acronym}[WSXXXX]
    \acro{LLM}[LLM]{Large Language Model}
    \acro{API}[API]{Application Programming Interface}
    \acro{REST}[REST]{Representational State Transfer}
    \acro{SQL}[SQL]{Structured Query Language}
    \acro{GPU}[GPU]{Graphics Processing Unit}
    \acro{SSL}[SSL]{Secure Sockets Layer}
    \acro{TLS}[TLS]{Transport Layer Security}
    \acro{OAuth}[OAuth]{Open Authorization}
    \acro{XSS}[XSS]{Cross-Site Scripting}
    \acro{ML}[ML]{Machine Learning}
    \acro{UI}[UI]{User Interface}
    \acro{HTTP}[HTTP]{Hypertext Transfer Protocol}
    \acro{JSON}[JSON]{JavaScript Object Notation}
    \acro{CPU}[CPU]{Central Processing Unit}
    \acro{RAM}[RAM]{Random Access Memory}
\end{acronym}

\addbibresource[label=refs]{bibliography.bib}
%\addbibresource[label=ownpubs]{ownpubs.bib}
%% -----------------------------------------------------------------
%%          _                                       _   
%%       __| | ___   ___ _   _ _ __ ___   ___ _ __ | |_ 
%%      / _` |/ _ \ / __| | | | '_ ` _ \ / _ \ '_ \| __|
%%     | (_| | (_) | (__| |_| | | | | | |  __/ | | | |_ 
%%      \__,_|\___/ \___|\__,_|_| |_| |_|\___|_| |_|\__|
%%                                                      
%% -----------------------------------------------------------------

\begin{document}
%% Title page, abstract, declaration etc.:
\frontmatter{}

%% List of code listings, if you are using package minted
%\listoflistings

%\pagenumbering{arabic}
%\printglossary[type=\acronymtype,title={\acrlistname},]
%\printglossary[type=\acronymtype]
%Glossaries
%\printglossary


%%%% Chapters
\chapter{Úvod do projektu kryptoanalýzy}

\section{Prehľad projektu}
Tento projekt predstavuje inovatívny prístup k riešeniu kryptografických problémov pomocou veľkých jazykových modelov (LLM - Large Language Models). Projekt kombinuje klasické kryptografické metódy s modernými technikami strojového učenia na vytvorenie efektívneho nástroja pre kryptoanalýzu.

\section{Základné charakteristiky systému}
Systém je navrhnutý ako webová aplikácia s komplexnou architektúrou, ktorá zahŕňa niekoľko kľúčových komponentov:

\begin{itemize}
    \item Webové rozhranie pre interakciu s používateľmi
    \item Backend server postavený na Flask frameworku
    \item Integrovaný jazykový model pre kryptoanalýzu
    \item PostgreSQL databáza pre ukladanie dát
    \item Systém autentifikácie a autorizácie
\end{itemize}

\section{Funkcionalita jazykového modelu}
\subsection{Základné schopnosti modelu}
Jazykový model implementovaný v systéme disponuje nasledujúcimi schopnosťami:

\begin{itemize}
    \item Automatická detekcia typu šifry
    \item Analýza frekvencie znakov
    \item Kontextová analýza textu
    \item Generovanie možných kľúčov
    \item Verifikácia dešifrovaného textu
\end{itemize}

\subsection{Pokročilé funkcie}
Model obsahuje aj pokročilé funkcie pre spracovanie textu:

\begin{itemize}
    \item Adaptívne učenie z úspešných dešifrovaní
    \item Viacjazyčná podpora (SK, EN, CZ)
    \item Štatistická analýza úspešnosti
    \item Optimalizácia výkonu pomocou GPU akcelerácie
    \item Automatická korekcia chýb
\end{itemize}

\section{Interakcia s používateľom}
Systém poskytuje intuitívne používateľské rozhranie s nasledujúcimi funkciami:

\begin{itemize}
    \item Nahrávanie šifrovaných textov
    \item Výber metódy dešifrovania
    \item Sledovanie procesu dešifrovania v reálnom čase
    \item Zobrazenie štatistík a analýz
    \item Export výsledkov v rôznych formátoch
\end{itemize}

\chapter{Technická implementácia}

\section{Architektúra systému}
\subsection{Serverová časť}
Backend systému je implementovaný v jazyku Python s využitím nasledujúcich technológií:

\begin{itemize}
    \item Flask 3.1.0 pre REST API
    \item PostgreSQL pre perzistenciu dát
    \item PyTorch 2.6.0 pre ML operácie
    \item Transformers 4.51.1 pre prácu s jazykovými modelmi
\end{itemize}

\subsection{Databázová schéma}
Systém využíva komplexnú databázovú schému s nasledujúcimi hlavnými tabuľkami:

\begin{itemize}
    \item Users - správa používateľov
    \item Ciphers - informácie o šifrách
    \item Models - správa modelov
    \item Decryption\_Attempts - záznamy o pokusoch
    \item Decryption\_Results - výsledky dešifrovania
    \item Manual\_Corrections - manuálne korekcie
\end{itemize}

\section{Bezpečnostné opatrenia}
Systém implementuje viacero bezpečnostných mechanizmov:

\begin{itemize}
    \item Hašovanie hesiel pomocou bcrypt
    \item OAuth 2.0 pre autentifikáciu
    \item SSL/TLS šifrovanie pre databázové spojenie
    \item Rate limiting pre API endpointy
    \item Ochrana proti SQL injection a XSS útokom
\end{itemize}

\chapter{Proces vývoja}

\section{Metodika vývoja}
Projekt bol vyvíjaný pomocou agilnej metodológie s nasledujúcimi charakteristikami:

\begin{itemize}
    \item Iteratívny prístup k vývoju
    \item Pravidelné aktualizácie a optimalizácie
    \item Kontinuálne testovanie
    \item Automatizované nasadenie
\end{itemize}

\section{História vývoja}
Vývoj projektu prebiehal v troch hlavných fázach:

\subsection{Fáza 1: Základná implementácia (Apríl 2025)}
\begin{itemize}
    \item Implementácia základného servera
    \item Vytvorenie používateľského rozhrania
    \item Integrácia autentifikácie
\end{itemize}

\subsection{Fáza 2: Vylepšenia a optimalizácia}
\begin{itemize}
    \item Optimalizácia výkonu servera
    \item Riešenie problémov s pamäťou
    \item Aktualizácia závislostí
\end{itemize}

\subsection{Fáza 3: Finálne úpravy}
\begin{itemize}
    \item Aktualizácia požiadaviek
    \item Finálne vylepšenia
    \item Optimalizácia výkonu
\end{itemize}

\chapter{Záver}

Projekt predstavuje komplexné riešenie v oblasti kryptoanalýzy, ktoré kombinuje klasické kryptografické metódy s modernými prístupmi strojového učenia. Systém poskytuje robustnú platformu pre výskum a praktické aplikácie v oblasti bezpečnosti a kryptografie.

Implementované riešenie demonštruje potenciál využitia veľkých jazykových modelov v oblasti kryptoanalýzy a otvára nové možnosti pre budúci výskum v tejto oblasti. Systém je navrhnutý s dôrazom na škálovateľnosť, bezpečnosť a používateľskú prívetivosť. 
\include{chapters/Chapter2}
\include{chapters/Chapter3}
\include{chapters/Chapter4}
\include{chapters/Chapter5}
\include{chapters/Chapter6}

%


% good linebraking of bibtex url
\setcounter{biburllcpenalty}{7000}
\setcounter{biburlucpenalty}{8000}

\footnotesize{
% The bibliography
\printbibliography[heading=bibintoc]
}


%\printbibliography
%\addcontentsline{toc}{chapter}{\numberline{}Zoznam vybranej publikačnej činnosti autora}
%\include{chapters/publikacie}
% List of acronyms
\printglossary[type=\acronymtype]
%\label{theend}  % the last page of the thesis
%% Appendix
%\include{appendixes/prilohy-front}
\include{appendixes/prilohy}
\appendix
\renewcommand\chaptername{Príloha}
%\chapter*{Zoznam príloh}
\include{appendixes/prilohaa}
\include{appendixes/prilohab}
%\include{appendixes/prilohac}
%\include{appendixes/prilohad}
%%







% zivotopis autora
%\curriculumvitae\protect
%Táto časť\/ je nepovinná. Autor tu môže uviesť\/ svoje biografické
%údaje, údaje o~záujmoch, účasti na~projektoch, účasti na~súťažiach,
%získané ocenenia, zahraničné pobyty na~praxi, domácu prax, publikácie
%a~pod.

\end{document}

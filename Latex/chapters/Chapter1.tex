\chapter{Úvod do projektu kryptoanalýzy}

\section{Prehľad projektu}
Tento projekt predstavuje inovatívny prístup k riešeniu kryptografických problémov pomocou veľkých jazykových modelov (LLM - Large Language Models). Projekt kombinuje klasické kryptografické metódy s modernými technikami strojového učenia na vytvorenie efektívneho nástroja pre kryptoanalýzu.

\section{Základné charakteristiky systému}
Systém je navrhnutý ako webová aplikácia s komplexnou architektúrou, ktorá zahŕňa niekoľko kľúčových komponentov:

\begin{itemize}
    \item Webové rozhranie pre interakciu s používateľmi
    \item Backend server postavený na Flask frameworku
    \item Integrovaný jazykový model pre kryptoanalýzu
    \item PostgreSQL databáza pre ukladanie dát
    \item Systém autentifikácie a autorizácie
\end{itemize}

\section{Funkcionalita jazykového modelu}
\subsection{Základné schopnosti modelu}
Jazykový model implementovaný v systéme disponuje nasledujúcimi schopnosťami:

\begin{itemize}
    \item Automatická detekcia typu šifry
    \item Analýza frekvencie znakov
    \item Kontextová analýza textu
    \item Generovanie možných kľúčov
    \item Verifikácia dešifrovaného textu
\end{itemize}

\subsection{Pokročilé funkcie}
Model obsahuje aj pokročilé funkcie pre spracovanie textu:

\begin{itemize}
    \item Adaptívne učenie z úspešných dešifrovaní
    \item Viacjazyčná podpora (SK, EN, CZ)
    \item Štatistická analýza úspešnosti
    \item Optimalizácia výkonu pomocou GPU akcelerácie
    \item Automatická korekcia chýb
\end{itemize}

\section{Interakcia s používateľom}
Systém poskytuje intuitívne používateľské rozhranie s nasledujúcimi funkciami:

\begin{itemize}
    \item Nahrávanie šifrovaných textov
    \item Výber metódy dešifrovania
    \item Sledovanie procesu dešifrovania v reálnom čase
    \item Zobrazenie štatistík a analýz
    \item Export výsledkov v rôznych formátoch
\end{itemize}

\chapter{Technická implementácia}

\section{Architektúra systému}
\subsection{Serverová časť}
Backend systému je implementovaný v jazyku Python s využitím nasledujúcich technológií:

\begin{itemize}
    \item Flask 3.1.0 pre REST API
    \item PostgreSQL pre perzistenciu dát
    \item PyTorch 2.6.0 pre ML operácie
    \item Transformers 4.51.1 pre prácu s jazykovými modelmi
\end{itemize}

\subsection{Databázová schéma}
Systém využíva komplexnú databázovú schému s nasledujúcimi hlavnými tabuľkami:

\begin{itemize}
    \item Users - správa používateľov
    \item Ciphers - informácie o šifrách
    \item Models - správa modelov
    \item Decryption\_Attempts - záznamy o pokusoch
    \item Decryption\_Results - výsledky dešifrovania
    \item Manual\_Corrections - manuálne korekcie
\end{itemize}

\section{Bezpečnostné opatrenia}
Systém implementuje viacero bezpečnostných mechanizmov:

\begin{itemize}
    \item Hašovanie hesiel pomocou bcrypt
    \item OAuth 2.0 pre autentifikáciu
    \item SSL/TLS šifrovanie pre databázové spojenie
    \item Rate limiting pre API endpointy
    \item Ochrana proti SQL injection a XSS útokom
\end{itemize}

\chapter{Proces vývoja}

\section{Metodika vývoja}
Projekt bol vyvíjaný pomocou agilnej metodológie s nasledujúcimi charakteristikami:

\begin{itemize}
    \item Iteratívny prístup k vývoju
    \item Pravidelné aktualizácie a optimalizácie
    \item Kontinuálne testovanie
    \item Automatizované nasadenie
\end{itemize}

\section{História vývoja}
Vývoj projektu prebiehal v troch hlavných fázach:

\subsection{Fáza 1: Základná implementácia (Apríl 2025)}
\begin{itemize}
    \item Implementácia základného servera
    \item Vytvorenie používateľského rozhrania
    \item Integrácia autentifikácie
\end{itemize}

\subsection{Fáza 2: Vylepšenia a optimalizácia}
\begin{itemize}
    \item Optimalizácia výkonu servera
    \item Riešenie problémov s pamäťou
    \item Aktualizácia závislostí
\end{itemize}

\subsection{Fáza 3: Finálne úpravy}
\begin{itemize}
    \item Aktualizácia požiadaviek
    \item Finálne vylepšenia
    \item Optimalizácia výkonu
\end{itemize}

\chapter{Záver}

Projekt predstavuje komplexné riešenie v oblasti kryptoanalýzy, ktoré kombinuje klasické kryptografické metódy s modernými prístupmi strojového učenia. Systém poskytuje robustnú platformu pre výskum a praktické aplikácie v oblasti bezpečnosti a kryptografie.

Implementované riešenie demonštruje potenciál využitia veľkých jazykových modelov v oblasti kryptoanalýzy a otvára nové možnosti pre budúci výskum v tejto oblasti. Systém je navrhnutý s dôrazom na škálovateľnosť, bezpečnosť a používateľskú prívetivosť. 